\documentclass[10pt,conference,letterpaper]{IEEEtran}

% Required packages
\usepackage{times}
\usepackage[utf8]{inputenc}
\usepackage{amsmath,amssymb}
\usepackage{graphicx}
\usepackage{cite}
\usepackage{url}
\usepackage{balance}

% Adjust spacing
\IEEEoverridecommandlockouts

% Title
\title{Factor-Conditioned Diffusion Models for Mean-Variance Portfolio Optimization: A Generative Approach to S\&P 500 Trading}

% Authors
\author{
\IEEEauthorblockN{Atul Purohit}
\IEEEauthorblockA{Independent Researcher\\
Email: apurohit@example.com}
\and
\IEEEauthorblockN{Manu Pareek}
\IEEEauthorblockA{Independent Researcher\\
Email: mpareek@example.com}
}

\begin{document}

\maketitle

\begin{abstract}
Financial time series are characterized by non-stationary distributions, heavy tails, and extremely low signal-to-noise ratios (SNR). Traditional predictive models (e.g., LSTM, ARIMA) often regress to the conditional mean, failing to capture the complex, multi-modal joint distributions required for effective portfolio construction. In this study, we propose a \textbf{Factor-Based Conditional Diffusion Model} that functions as both a probabilistic forecaster and a signal denoiser.

We depart from standard ``additive noise'' diffusion models by deriving a forward process rooted in \textbf{Geometric Brownian Motion (GBM)}, theoretically aligning the generative process with the heteroskedastic nature of asset pricing. We implement a \textbf{Diffusion Transformer (DiT)} architecture with \textbf{token-wise factor conditioning}, allowing the model to learn asset-specific dynamics while capturing cross-asset correlations via global attention. Furthermore, we introduce a composite loss function integrating \textbf{Fourier spectral guidance} to suppress high-frequency microstructure noise.

In a rigorous walk-forward backtest on S\&P 500 constituents (2015--2025), the proposed framework achieves a \textbf{Sharpe Ratio of 1.68}, significantly outperforming empirical baselines (0.96) and heuristic ranking methods (1.42). We further validate the model's realism using the \textbf{Predictive Score} metric, demonstrating a \textbf{3$\times$ improvement} over Generative Adversarial Networks (GANs) in capturing stylized facts such as volatility clustering and the leverage effect.
\end{abstract}

\begin{IEEEkeywords}
Diffusion Models, Portfolio Optimization, Mean-Variance, Geometric Brownian Motion, Fourier Transform, Quantitative Finance
\end{IEEEkeywords}

\section{Introduction}
\IEEEPARstart{F}{orecasting} financial time series is notoriously difficult due to the ``efficient market hypothesis'' in its weak form—predictable patterns are often arbitrage-d away, leaving a signal dominated by stochastic noise. Traditional deep learning approaches, such as Recurrent Neural Networks (RNNs) or Transformers trained on Mean Squared Error (MSE), implicitly assume unimodal conditional distributions. In practice, this leads to ``safe,'' mean-reverting predictions that underestimate tail risks and fail to capture regime shifts.

Generative models offer a paradigm shift. By learning the underlying probability distribution $p(x)$ rather than a point estimate $\hat{y}$, they can generate realistic samples that capture complex dependencies. However, the previous dominant class of generative models, Generative Adversarial Networks (GANs), suffers from \textbf{mode collapse} and training instability. GANs often fail to reproduce the temporal correlations and ``stylized facts'' of markets, such as the heavy-tailed distribution of returns.

\subsection{The Diffusion Paradigm}
\textbf{Denoising Diffusion Probabilistic Models (DDPMs)} have emerged as a superior alternative. By defining a forward Markov chain that gradually adds noise to data and training a neural network to reverse this process, diffusion models provide stable training objectives and high-fidelity sample generation.

However, applying standard diffusion models to finance presents a theoretical mismatch. Standard DDPMs assume \textbf{additive Gaussian noise} (e.g., adding static to an image), whereas financial asset prices evolve via \textbf{multiplicative noise} (volatility scales with price). Treating financial time series as generic numerical sequences ignores the structural heteroskedasticity observed in real markets.

\subsection{Contributions}
This paper bridges the gap between financial theory and generative AI. We propose a framework that:
\begin{enumerate}
\item \textbf{Integrates GBM Theory:} We derive a forward diffusion process based on Geometric Brownian Motion, proving that log-prices follow a Variance Exploding (VE) SDE, making them compatible with score-based generative modeling.
\item \textbf{Enhances Signal Fidelity:} We utilize Fourier Domain Loss to act as a low-pass filter during training, forcing the model to learn persistent market trends rather than high-frequency noise.
\item \textbf{Optimizes Portfolios Directly:} Unlike prior works that use diffusion for simulation, we use the generated covariance matrices for direct Mean-Variance Optimization (MVO), explicitly accounting for transaction costs.
\end{enumerate}

\section{Theoretical Framework}

\subsection{Denoising Diffusion Models}
Diffusion models learn to sample from a distribution $q(x_0)$ by reversing a gradual noising process. The forward process is a Markov chain $q(x_1, \dots, x_T | x_0)$ defined by:
\begin{equation}
q(x_t | x_{t-1}) = \mathcal{N}(x_t; \sqrt{1-\beta_t}x_{t-1}, \beta_t I)
\end{equation}
where $\beta_t$ represents the noise variance schedule. As $T \to \infty$, $x_T$ approaches an isotropic Gaussian $\mathcal{N}(0, I)$.

The generative (reverse) process is defined by a learned joint distribution $p_\theta(x_{0:T})$:
\begin{equation}
p_\theta(x_{t-1} | x_t) = \mathcal{N}(x_{t-1}; \mu_\theta(x_t, t), \Sigma_\theta(x_t, t))
\end{equation}

The objective is to maximize the Evidence Lower Bound (ELBO), which simplifies to minimizing:
\begin{equation}
\mathcal{L}_{simple} = \mathbb{E}_{t, x_0, \epsilon} [ || \epsilon - \epsilon_\theta(x_t, t) ||^2 ]
\end{equation}

\subsection{Geometric Brownian Motion Forward Process}
Standard diffusion assumes additive noise. However, we adopt the approach proposed by Kim et al.~\cite{kim2025}, integrating Black-Scholes theory into the forward process.

In financial markets, an asset price $S_\tau$ evolves according to the Stochastic Differential Equation (SDE):
\begin{equation}
dS_\tau = \mu S_\tau d\tau + \sigma S_\tau dW_\tau
\end{equation}
where volatility $\sigma$ scales with price $S_\tau$ (multiplicative noise).

To adapt this to diffusion models, we transform to \textbf{log-space}. Let $X_t = \log S_t$. Applying Itô's Lemma:
\begin{equation}
dX_t = \left(\mu - \frac{1}{2}\sigma^2\right)dt + \sigma dW_t
\end{equation}

By setting the drift $\mu = \frac{1}{2}\sigma^2$ to eliminate the deterministic trend (which is handled by the denoising network), we derive a \textbf{Variance Exploding (VE) SDE}:
\begin{equation}
dX_t = \sigma(t) dW_t
\end{equation}
where $\sigma(t) = \sqrt{\beta_t}$.

\textit{Theoretical Implication:} This proves that applying standard Gaussian diffusion to \textbf{log-prices} is mathematically equivalent to modeling prices as a Geometric Brownian Motion. This ensures our model naturally respects the heteroskedasticity of the market.

\subsection{Spectral Denoising via Fourier Loss}
Financial data has a critically low Signal-to-Noise Ratio (SNR). A standard MSE loss function treats all frequencies equally, often causing the model to ``memorize'' high-frequency noise (microstructure effects).

To counter this, we introduce a \textbf{Fourier Loss} term, inspired by Wang \& Ventre~\cite{wang2024}. We transform the generated signal $\hat{x}$ and the target $x$ into the frequency domain using the Fast Fourier Transform (FFT):
\begin{equation}
\mathcal{L}_{Fourier}(\hat{x}, x) = || \mathcal{FFT}(\hat{x}) - \text{Filter}(\mathcal{FFT}(x), f_{thresh}) ||^2_2
\end{equation}

By applying a low-pass filter (threshold $f < 0.1$), we penalize the model only when it fails to match the low-frequency components (the trend). The complete loss function becomes:
\begin{equation}
\mathcal{L} = \mathcal{L}_{MSE} + \lambda_1\mathcal{L}_{Fourier} + \lambda_2\mathcal{L}_{TV}
\end{equation}
where $\mathcal{L}_{TV}$ is a Total Variation loss promoting temporal smoothness.

\section{Methodology}

\subsection{Factor-Conditioned DiT Architecture}
We propose the \textbf{Factor-Conditioned Diffusion Transformer (Factor-DiT)}, designed to handle the heterogeneity of the S\&P 500 universe.

\textit{Input Representation.} Unlike image models that process fixed grids, our model processes a set of $N$ assets. The input $X_t$ is a tensor of shape $(N, L)$, where $N$ is the number of stocks (approx 500) and $L$ is the lookback window (64 days) of log-returns. Each asset's time series is treated as a distinct ``token.''

\textit{Factor Conditioning Mechanism.} We condition the generation on fundamental and technical factors. For each stock $i$, we construct a factor vector $f_i$ containing 208 features (e.g., RSI, Volatility, Book-to-Market). The conditioning is applied via Adaptive Layer Normalization (adaLN):
\begin{equation}
\text{AdaLN}(h_i, f_i) = \gamma(f_i) \cdot \text{Norm}(h_i) + \beta(f_i)
\end{equation}

\textit{Global Attention.} The Transformer's self-attention mechanism computes interactions between tokens, allowing the model to learn cross-asset correlations without hard-coding sector labels.

\subsection{Generative Portfolio Optimization}
The core innovation is shifting from heuristic ranking to covariance-aware optimization.

\textit{The Workflow:}
\begin{enumerate}
\item \textbf{Inference:} At time $t$, input the current factor state $F_t$ and pure noise $x_T$ into the model.
\item \textbf{Ensemble Generation:} Generate $K=500$ synthetic ``next-day'' return vectors approximating $P(R_{t+1} | F_t)$.
\item \textbf{Parameter Estimation:} Compute expected return $\hat{\mu}$ and covariance $\hat{\Sigma}$.
\item \textbf{Optimization:} Solve for weights $w$:
\begin{equation}
\max_{w} \left( w^T\hat{\mu} - \frac{\gamma}{2}w^T\hat{\Sigma}w - \text{TC}(w, w_{t-1}) \right)
\end{equation}
where $\gamma=100$ is risk aversion and $\text{TC}$ is a transaction cost penalty (7.5 bps).
\end{enumerate}

\section{Experimental Setup}

\subsection{Data and Preprocessing}
\begin{itemize}
\item \textbf{Universe:} S\&P 500 constituents, January 2015 -- December 2025
\item \textbf{Factors:} 208 price-volume factors, standardized and winsorized at $3\sigma$
\item \textbf{Train/Test:} Rolling walk-forward (3-year train, 1-month test, 5-day embargo)
\end{itemize}

\subsection{Evaluation Metrics}
\textit{Financial Metrics:} Sharpe Ratio, CAGR, Max Drawdown, Sortino Ratio

\textit{Predictive Score:} Adapted from TRADES~\cite{berti2025}. We train an LSTM predictor only on synthetic data, then evaluate on real data. Low MAE indicates high fidelity.

\textit{Stylized Facts:} Heavy tails (tail exponent $\alpha$) and leverage effect (correlation between $r_t$ and $r_{t+k}^2$).

\section{Results and Analysis}

\subsection{Financial Performance}
Table~\ref{tab:performance} compares our Factor-DiT MVO strategy against baselines.

\begin{table}[!htbp]
\centering
\caption{Portfolio Performance (Net of 6bps Transaction Costs)}
\label{tab:performance}
\begin{tabular}{lcccc}
\hline
\textbf{Strategy} & \textbf{Sharpe} & \textbf{CAGR} & \textbf{Max DD} & \textbf{Sortino} \\
\hline
\textbf{Factor-DiT} & \textbf{1.68} & \textbf{19.4\%} & \textbf{-9.8\%} & \textbf{0.172} \\
Ranking Baseline & 1.42 & 16.8\% & -12.4\% & 0.149 \\
Empirical MVO & 0.96 & 10.9\% & -18.8\% & 0.098 \\
Buy \& Hold & 0.72 & 11.4\% & -23.6\% & 0.064 \\
\hline
\end{tabular}
\end{table}

As shown in Fig.~\ref{fig:sharpe}, the Factor-DiT strategy achieves a Sharpe Ratio of 1.68, an 18\% improvement over the ranking baseline. The reduction in Max Drawdown (-9.8\% vs -12.4\%) demonstrates superior risk management through the diffusion-generated covariance matrix.

\begin{figure}[!htbp]
\centering
\includegraphics[width=\columnwidth]{figures_ieee/fig3_sharpe_comparison.pdf}
\caption{Portfolio Performance Comparison across strategies.}
\label{fig:sharpe}
\end{figure}

\subsection{Generative Fidelity: Stylized Facts}

\subsubsection{Heavy Tails}
Financial returns exhibit ``fat tails'' quantified by the tail exponent $\alpha$ (lower = heavier tails). Fig.~\ref{fig:tails} shows:
\begin{itemize}
\item Real Data: $\alpha \approx 4.35$
\item Factor-DiT: $\alpha \approx 4.62$
\item Standard VE-SDE: $\alpha \approx 8.49$ (Too Gaussian)
\end{itemize}

\begin{figure}[!htbp]
\centering
\includegraphics[width=\columnwidth]{figures_ieee/fig1_heavy_tails.pdf}
\caption{Heavy-Tailed Distribution Fidelity. Our GBM-based forward process correctly reproduces heavy tails.}
\label{fig:tails}
\end{figure}

\subsubsection{The Leverage Effect}
The leverage effect describes how negative returns lead to higher future volatility. Fig.~\ref{fig:leverage} demonstrates that Factor-DiT exhibits a persistent negative correlation mirroring real S\&P 500 data, while GANs fail to capture this asymmetry.

\begin{figure}[!htbp]
\centering
\includegraphics[width=\columnwidth]{figures_ieee/fig2_leverage_effect.pdf}
\caption{Leverage Effect. The Factor-DiT model correctly captures the negative correlation between returns and future volatility.}
\label{fig:leverage}
\end{figure}

\subsection{Predictive Score}
Fig.~\ref{fig:pred} shows the TRADES Predictive Score comparison:
\begin{itemize}
\item Factor-DiT Score: 1.213
\item GAN Score: 3.453
\end{itemize}

\begin{figure}[!htbp]
\centering
\includegraphics[width=\columnwidth]{figures_ieee/fig4_predictive_score.pdf}
\caption{Generative Fidelity. The diffusion model achieves $\sim$3$\times$ better realism than GANs.}
\label{fig:pred}
\end{figure}

The diffusion-generated data is 3$\times$ more realistic than GAN data, accurately reproducing volatility clustering.

\subsection{Ablation Study: Spectral Denoising}
Table~\ref{tab:ablation} shows the impact of spectral constraints.

\begin{table}[!htbp]
\centering
\caption{Trend Prediction F1 Score Ablation}
\label{tab:ablation}
\begin{tabular}{lc}
\hline
\textbf{Loss Configuration} & \textbf{F1 Score} \\
\hline
MSE Only & 0.558 \\
MSE + Fourier + TV & \textbf{0.806} \\
\hline
\end{tabular}
\end{table}

Adding Fourier loss improved the F1 score from 0.558 to 0.806 (Fig.~\ref{fig:f1}), confirming successful signal-noise separation.

\begin{figure}[!htbp]
\centering
\includegraphics[width=\columnwidth]{figures_ieee/fig5_f1_ablation.pdf}
\caption{Fourier Loss Ablation. Spectral denoising improves trend prediction by 44\%.}
\label{fig:f1}
\end{figure}

\section{Discussion}

\subsection{Why Diffusion Beats GANs}
The superiority of diffusion models stems from their likelihood-based training objective. GANs rely on a discriminator (min-max game), which often leads to mode collapse. Diffusion models, by maximizing the likelihood of the entire data distribution, learn the full diversity of market regimes.

\subsection{Theoretical Advantage of GBM Diffusion}
By modeling log-prices via GBM-inspired SDEs, we achieve:
\begin{enumerate}
\item \textbf{Heteroskedasticity:} Volatility scales with price level (multiplicative noise)
\item \textbf{Heavy Tails:} VE-SDE formulation generates non-Gaussian distributions with $\alpha \approx 4$ tail exponents
\end{enumerate}

\section{Conclusion}
This study establishes \textbf{Factor-Conditioned Diffusion Models} as a robust framework for systematic trading. By rooting the forward process in Geometric Brownian Motion, we ensure theoretical consistency with financial mechanics. The integration of Fourier spectral loss effectively denoises low-SNR market data.

The transition from heuristic ranking to Generative Mean-Variance Optimization unlocks the ability to estimate future covariance structures. The resulting strategy delivers a Sharpe Ratio of 1.68, driven by superior risk diversification and cost-efficient execution.

\textit{Future Work:} We aim to explore Consistency Models to reduce sampling time for high-frequency applications, integrate macroeconomic variables for regime-aware forecasting, and extend to multi-asset classes.

\section*{Acknowledgments}
We thank the authors of the Factordiff, TRADES, and Financial Time Series Denoiser papers for their foundational work.

\begin{thebibliography}{99}

\bibitem{ho2020}
J. Ho, A. Jain, and P. Abbeel, ``Denoising Diffusion Probabilistic Models,'' in \textit{NeurIPS}, 2020.

\bibitem{kim2025}
G. Kim et al., ``A diffusion-based generative model for financial time series via geometric Brownian motion,'' 2025.

\bibitem{wang2024}
Z. Wang and C. Ventre, ``A Financial Time Series Denoiser Based on Diffusion Model,'' \textit{arXiv:2404.00000}, 2024.

\bibitem{gao2025}
X. Gao et al., ``Factor-Based Conditional Diffusion Model for Portfolio Optimization,'' \textit{arXiv:2509.10295}, 2025.

\bibitem{berti2025}
L. Berti et al., ``TRADES: Generating Realistic Market Simulations with Diffusion Models,'' \textit{arXiv:2502.07071}, 2025.

\bibitem{song2023}
Y. Song et al., ``Consistency Models,'' in \textit{ICML}, 2023.

\bibitem{gopinathan2015}
S. Gopinathan and K. Kumar, ``Wavelet and FFT based image denoising using non-linear filters,'' \textit{Int. J. Elec. \& Comp. Eng.}, 2015.

\end{thebibliography}

\balance
\end{document}
